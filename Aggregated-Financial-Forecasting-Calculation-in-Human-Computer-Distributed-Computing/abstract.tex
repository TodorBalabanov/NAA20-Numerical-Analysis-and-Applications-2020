\documentclass[12pt,a4paper]{article} 

\title{\bf Aggregated Financial Forecasting Calculation in Human-Computer Distributed Computing}

\author{Petar Tomov, Iliyan Zankinski, Todor Balabanov\textsuperscript{\tiny 0000-0003-3139-069X} \\ Institute of Information and Communication Technologies \\ Bulgarian Academy of Sciences \\ acad. Georgi Bonchev Str., block 2, 1113 Sofia, Bulgaria \\ http://www.iict.bas.bg/}

\date{} 

\begin{document} 

\maketitle 

Three of the most famous financial forecasting donated distributed computing projects for the last decade were MoneyBee, GStock and MQL5 Cloud Network. MoneyBee project was organized as donated distributed computing screen-saver. Artificial neural networks were trained to forecast financial time series during desktop computers idle usage. GStock project used donated distributed computing for trading strategies testing. Many different investment strategies were checked over historical data and the best suited were applied for buy/sell signals generation. MQL5 Cloud Network is paid distributed computing platform in which trading strategies can be tested. 

The idea of a desktop computer screen-saver used for artificial neural networks training was extended in VitoshaTrade project where Android Active Wallpaper is used for the same purpose. The project implementation goes on in IICT-BAS and the forecasting capabilities are evolved in the direction of human-computer based distributed computing. The users are asked to vote for the future change in the price of Forex currency pairs. Android operating system provides the ideal user interface capabilities for such voting by its widget components. By HTTP protocol the vote of the users is collected on a remote PHP-MySQL based server. As proposed in NSFDE\&A'20, users' votes are classified in four categories: 1) High voting frequency and high guess rate users; 2) Low voting frequency and high guess rate users; 3) High voting frequency and low guess rate users; 4) Low voting frequency and low guess rate users.

This research proposes financial forecasting calculation based on users classification and votes given for particular financial future event. 

\end{document}